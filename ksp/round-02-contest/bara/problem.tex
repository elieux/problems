\documentclass{article}
\usepackage{ucs}
\usepackage[utf8x]{inputenc}
\usepackage{a4wide}
\usepackage[english]{babel}
\usepackage{amsmath}
\usepackage{amssymb}
\usepackage{units}

%renewcommand{\theenumi}{\Alph{enumi}}

\begin{document}

\section*{\Huge Bára a OCD}
Bára ráda kupuje dobré džusy. Myslí si, že nejdražší džusy jsou nejlepší. Když jde Bára do obchodu, kupuje si hned několik ($k$) džusů, aby jí nějakou dobu vydržely. Naneštěstí má Bára OCD (obsedantně kompulzivní porucha) a může si kupovat pouze džusy, které stojí v regálu vedle sebe. Takže když jde do obchodu a chce si koupit džus, tak si bere $k$ vedle sebe stojících džusů, i kdyby to nebyly zrovna ty úplně nejdražší. Bára se rozhodla, že nejlepší výběr džusů je takový, v němž džusy dohromady budou stát nejvíce ze všech možných výběrů. Pomožte Báře spočítat, kolik utrátí za džusy.

\section*{Vstup a výstup}
První řádek uvádí celkový počet $N$ ($\leq 500$) testovacích vstupů. Následuje $N$ testovacích vstupů popsaných následovně: první řádek vstupu uvádí celkový počet $n$ ($2 \leq n \leq 50000$) džusu v regálu a počet džusů $k$, které Bára chce koupit ($1 \leq k \leq n$). Na dalším řádku jsou uvedeny ceny džusů v tom pořadí, v jakém jsou v regálu.

Vaším ukolem je pro každý testovací vstup vypsat největší částku, kterou Bára může utrátit, chce-li koupit $k$ jdoucích po sobě džusů.

\section*{Vzorový vstup}
\begin{verbatim}
2
3 3
1 2 3
4 2
1 9 3 4
\end{verbatim}

\section*{Vzorový výstup}
\begin{verbatim}
6
12
\end{verbatim}

\end{document}
