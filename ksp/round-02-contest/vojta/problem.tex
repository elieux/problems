\documentclass{article}
\usepackage{ucs}
\usepackage[utf8x]{inputenc}
\usepackage{a4wide}
\usepackage[english]{babel}
\usepackage{amsmath}
\usepackage{amssymb}
\usepackage{units}

%renewcommand{\theenumi}{\Alph{enumi}}

\begin{document}

\section*{\Huge Vojta a fejsbůk}
Fejsbůk předevčírem představil novou fičůru - propagování viditelností příspěvků na základě přátelství. Funguje to tak: Adam vidí všechno, co si nahraje na fejsbůk. Taktéž to vidí jeho kamarádi. Na to samé se můžou podívat kamarádi těchto kamarádů. A kamarádi kamarádů kamarádů. Zkrátka přatelí-li se X s Y, X vidí všechno, co vidí Y.

Vojta rovněž předevčírem nahrál na fejsbůk fotky z poslední akce, ale nevěděl přitom, že vlastně sdílí své fotky s o něco větším množstvím lidí, než zamýšlel. Po tom, co se jeho rodiče (se kterými se na fejsbůku "nekamarádí") začali u večeře na onu akci vyptávat, Vojta se vylekal a fotky stáhl. Stále ho ale trápí, kolik lidí ty fotky mohlo vidět, a proto se obrací na vás pro pomoc.

\section*{Vstup a výstup}
První řádek uvádí celkový počet $N$ testovacích vstupů. Následuje $N$ testovacích vstupů popsaných následovně: první řádek vstupu uvádí celkový počet $n$ ($1 \leq n \leq 1000$) lidí v sociální síti a číselný identifikátor $m$ Vojty ($0 \leq m < n$). Dále v jednotlivých řádcích následují dvojice čísel indikujících, že uživatel s identifikátorem $i$ kamarádí s uživatelem s identifikátorem $j$ ($0 \leq i, j < n$, $i \neq j$). Poslední řádek testovacího vstupu jsou dvě nuly (ale neznamená to, že uživatel 0 přátelí s uživatelem 0).

Vaším úkolem je pro každý testovací vstup vypsat počet uživatelů (mimo Vojtu), kteří mohli vidět Vojtovy fotky.

\section*{Vzorový vstup}
\begin{verbatim}
2
10 5
1 2
3 4
4 5
5 6
6 7
8 9
0 0
1000 5
0 0
\end{verbatim}

\section*{Vzorový výstup}
\begin{verbatim}
4
0
\end{verbatim}

\end{document}
